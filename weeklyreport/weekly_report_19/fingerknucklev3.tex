\subsection{Within-Database Experiments}
\subsubsection{Finger Knuckle V3 Database with Deformation}

The Finger Knuckle V3 Database have 1-104 subjects that have two session samples, and the rest subjects of first session 105-221 just offer one session samples. So as the first experiment, I firstly fine-tuned my model on the one-session 1-104 dataset, and test on the another session 1-104 subjects. So it will have $104*6=624$ genuine matching scores, and have $104*103*6=64272$ imposter matching scores. From the below figure, if the false accept rate is below $10^{-2}$, the RFN-128-WRS is better than the RFN-128-WS. 

\begin{figure}[H]
	\centering
	\begin{subfigure}[b]{0.6\linewidth}
		\includegraphics[width=\linewidth]{Figures/ft/fkv3-roc-ibrs.png}
	\end{subfigure}
\end{figure}



\begin{table}[H]
    \centering
    \begin{tabular}{ccccl}
    \cline{1-4}
    Model            & Genuine & Imposter & EER     &  \\ \cline{1-4}
    RFN-WRS          & 624    & 64272   & 0.08014 &  \\
    DeConvRFN-WRS    & 624    & 64272   & 0.09296 &  \\
    EfficientNet-WRS & 624    & 64272   & 0.09377 &  \\
    RFN-WS           & 624    & 64272   & 0.06732 &  \\ 
    RFN-IBRS           & 624    & 64272   & 0.09313 &  \\\cline{1-4}
    \end{tabular}
\end{table}

\textcolor{red}{I also used the Spatial Transformer Net Module (Jaderberg, Max, Karen Simonyan, and Andrew Zisserman. "Spatial transformer networks." Advances in neural information processing systems 28 (2015).), called STNet to actively spatially transform feature map. At present, the RFNet can get very high performance on finger knuckle database, then I add the STNet module at the end of RFNet to learn the homographic transformation matrix. Meanwhile, at the input side of the RFNet, I firstly split the raw images to several image blocks, and use different learnable kernel to learn image blocks feature. I called it ad ImageBlocksRFNet. However, both of them are not outperform the RFNet with the same evaluation protocol.}

\begin{figure}[H]
	\centering
	\begin{subfigure}[b]{0.6\linewidth}
		\includegraphics[width=\linewidth]{Figures/fkv3-protocol3_roc.pdf}
	\end{subfigure}
    \caption{\textcolor{red}{Compare peformance on the Finger Knuckle V3 database. The ImageBlocksRFNet and RFNWithSTNet are based on the RFNet.}}
\end{figure}



\begin{table}[H]
    \centering
    \begin{tabular}{ccccl}
    \cline{1-4}
    \textcolor{red}{Model}            & \textcolor{red}{Genuine} & \textcolor{red}{Imposter} & \textcolor{red}{EER}     &  \\ \cline{1-4}
    \textcolor{red}{ImageBlocksRFNet-IBRS}          & \textcolor{red}{624}    & \textcolor{red}{64272}   & \textcolor{red}{0.14938} &  \\
    \textcolor{red}{RFN-IBRS}          & \textcolor{red}{624}    & \textcolor{red}{64272}   & \textcolor{red}{0.09313} &  \\
    \textcolor{red}{RFNWithSTNet-WRS}          & \textcolor{red}{624}    & \textcolor{red}{64272}   & \textcolor{red}{0.09613} &  \\
    \textcolor{red}{RFN-WRS}          & \textcolor{red}{624}    & \textcolor{red}{64272}   & \textcolor{red}{0.08014} &  \\
    \textcolor{red}{RFN-WS}          & \textcolor{red}{624}    & \textcolor{red}{64272}   & \textcolor{red}{0.06732} &  \\
    \textcolor{red}{EfficientNet-WS}          & \textcolor{red}{624}    & \textcolor{red}{64272}   & \textcolor{red}{0.12341} &  \\\cline{1-4}
    \end{tabular}
\end{table}


As for the two-session protocol on the database. I should fine-tune my model on the 105-221 subjects, and use two-session protocol to evaluate my model performance on the 1-104 subjects dataset. In totally, it will generate $104*6=624$ genuine scores, and $104*103*6$ imposter scores.

\begin{figure}[H]
    \centering
    \includegraphics[width=0.6\linewidth]{Figures/two-session-fkv3-ibrs.png}
\end{figure}

\begin{table}[H]
    \centering
    \begin{tabular}{ccccl}
    \cline{1-4}
    Model                & Genuine              & Imposter             & EER                  &  \\ \cline{1-4}
    RFN-WS               & 624                 & 64272               & 0.25802              &  \\
    RFN-WRS              & 624                 & 64272               & 0.26924              &  \\
    RFN-IBRS           & 624                   & 64272               &0.23949 & \\ \cline{1-4}
    \multicolumn{1}{l}{} & \multicolumn{1}{l}{} & \multicolumn{1}{l}{} & \multicolumn{1}{l}{} & 
    \end{tabular}
\end{table}

The two-session protocol will use the session1 as the probe and use the session2 as the enrollment. As for the genuine matching scores, each sample of a subject will choose the minimal matching score when compare to the rest samples. In this kind of situation, it will have $104x6$ genuine matching scores. Meanwhile, as for the imposter matching scores, it will also choose the minimal value result in $104*103*6$ imposter matching scores on the confusion matrix.