\subsubsection{Index Finger Knuckle of Hand Dorsal Image Database}
\begin{figure}[H]
	\centering
		\includegraphics[width=0.6\linewidth]{Figures/hd_roc_compare.jpg}
\end{figure}

\begin{table}[H]
    \centering
    \begin{tabular}{ccccl}
    \cline{1-4}
    Model            & Genuine & Imposter & EER     &  \\ \cline{1-4}
    RFN-WRS          & 2848    & 2024928   & 0.02001 &  \\
    RFN-WS           & 2848    & 2024928   & 0.01650 &  \\ \cline{1-4}
    \end{tabular}
\end{table}

As for the experiment, the dataset totally contains 712 subjects, and I use the segmented Index finger knuckle as my dataset. And I fine-tuned my model on the first sample of each subject, and then use the rest four sample as the testing dataset. At the testing process, it has $712*4=2848$ genuine matching scores, and has $712*711*4=2024928$ imposter matching scores. The performance of RFN-128-WRS and RFN-128-WS is similar, but the RFN-128-WS is slightly better than RFN-128-WRS depend on the EER value.\textcolor{red}{ And we can get an information that the RFNet is better than the rest network in the ROC figure, including the FKNet.}